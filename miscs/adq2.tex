% Include preambule
\input{shared/udem-pre}
% Load Package
% a third test
% a 5 test
\usepackage[margin=1in]{geometry} % for margin
\usepackage{blindtext}
\usepackage{graphicx}
\usepackage{multicol}
\usepackage{tabularx}
\usepackage{booktabs}
\usepackage{pgffor}
\usepackage{color}
\usepackage[hyperfootnotes=true,colorlinks=true,allcolors={blue}]{hyperref}
\usepackage{ifthen}
\usepackage{indentfirst}
\usepackage{dcolumn}

\usepackage{nameref}
\usepackage{placeins} % put this in your pre-amble
\usepackage{flafter}  % put this in your pre-amble
\usepackage{float} % to manage figure float
\usepackage{authblk} % for co-author
\usepackage{pdfpages}
\usepackage{listings} % for formating code
\usepackage{dirtytalk} % for inserting quotation
% Control footnotes spacing
%\usepackage{footnote}
%\usepackage[multiple]{footmisc}
\setlength{\footnotesep}{0.3cm}%
\setlength{\skip\footins}{0.7cm}

\usepackage{bookmark}
%\usepackage{lmodern}
%\usepackage{mathptmx} % for nice font
\fontfamily{cmss}

\hypersetup{
  colorlinks=true,
  allcolors=blue
}

\usepackage{threeparttable}
\usepackage{booktabs, longtable}
\usepackage{tabularx}
\usepackage[singlelinecheck=false]{caption}
% for source under image
\newcommand{\source}[1]{\caption*{Source: {#1}} }

% set space interligne etc
\usepackage{setspace}
\onehalfspacing




%\usepackage{amssymb} %maths
%\usepackage{amsmath} %maths
\usepackage{siunitx} % using plus-minus instead of std
\usepackage[utf8]{inputenc} %useful to type directly diacritic characters
\usepackage[autostyle]{csquotes}
\DeclareUnicodeCharacter{2212}{-} % Unicode U+2212 is MINUS SIGN, which is not set up by default with inputenc
\DeclareUnicodeCharacter{2009}{~} % Unicode U+2212 is MINUS SIGN, which is not set up by default with inputenc{00A0}{~}

\DeclareUnicodeCharacter{0080}{+} % Unicode U+2212 is MINUS SIGN, which is not set up by default with inputenc{00A0}{~}

\DeclareUnicodeCharacter{0301}{-} % Unicode U+2212 is MINUS SIGN, which is not set up by default with inputenc{00A0}{~}

\DeclareUnicodeCharacter{0300}{-} % Unicode U+2212 is MINUS SIGN, which is not set up by default with inputenc{00A0}{~}

\DeclareUnicodeCharacter{001B}{-} % Unicode U+2212 is MINUS SIGN, which is not set up by default with inputenc{00A0}{~}

% Count of words
\usepackage{moreverb} % for verbatim ouput
\immediate\write18{texcount  -inc -merge -nobib -sum -sub=section \jobname.tex > ./texcount.tex}
\newcommand\wordcount{\verbatiminput{./texcount.tex}}

% normal header stat here


\title{Vieillissement démographique au Canada: Défis et Oppourtinités}
\author[1]{Gilbert MONTCHO}
\author[3]{Julien NAVAUX}
\author[1]{Yves CARRIERE}
\author[2]{Marcel MERETTE}



\affil[1]{Départment de Démographie, Université de Montréal}
\affil[2]{Faculté des sciences sociales, Université d'Ottawa}
\affil[3]{Chaire de recherche sur les enjeux économiques intergénérationnels, HEC Montréal}
\date{}
%%\orientation{orientation}%Ce champ est optionnel
\begin{document}
\maketitle

%\setstretch{1.3}
\section*{Résumé}



Bien qu'au Canada plus de 60\% des immigrants sont sélectionnés en fonction des besoins du marché du travail, la contribution des immigrants ne se limite pas à répondre à ces besoins. Entre autres, les immigrants reçoivent des transferts publics d'une part et ils contribuent aux recettes de l’État  d’autre part. Il est donc important d'étendre les débats sur l'immigration à d'autres aspects économiques comme l'impact fiscal. Les transferts incluent certaines dépenses de consommation individuelle comme, par exemple, les allocations familiales, les prestations de chômages, les pensions de retraite; et des dépenses de consommations collectives, dont la défense publique, la police et la justice, parmi tant d’autres. Les contributions représentent les impôts sur le revenu et les taxes sur la consommation. Le solde des ces transactions, c'est à dire la difference entre les coûts et contributions, pour un immigrants moyen par rapport à un non-immigrant représente l'effet de l'immigration sur les finances publiques.


Il s'agira dans ce troisième article d'estimer les coûts et contributions qu'un individu occasionne à l’État sur l'ensemble du cycle de vie, tout en tenant compte du statut d'immigration. Cet article permettra de répondre aux questions comme: l'immigrant coûte-t-il plus ou moins cher à l'État comparé à un non-immigrant et inclura

\vspace{0.7em}\par
Cette présentation discute les implications politiques du vieillissement démographique en prenant appui sur les résultats des études récentes sur l’impact fiscal de l’immigration, les tendances de la durée de vie au travail, et le potentiel des personnes sous-employées. Les résultats suggèrent que le vieillissement démographique présente autant d'oppourtinités d’adaptation sociale et technologique que de défis économiques.


%\newpage
\section*{Détails de la communication}

\begin{enumerate}
  \item Auteur et co-auteurs

  Gilbert MONTCHO,
  \href{mailto:gilbert.montcho@umontreal.ca}{gilbert.montcho@umontreal.ca}

  Yves CARRIERE,
  \href{mailto:yves.carriere@umontreal.ca}{yves.carriere@umontreal.ca}

  Julien NAVAUX,
  \href{mailto:julien.navaux@hec.ca}{julien.navaux@hec.ca}

  Marcel MERETTE
  \href{mailto:mmerette@uottawa.ca}{mmerette@uottawa.ca}

  \item Titre (voir plus haut)
  \item Résumé court (voir plus haut)
  \item Nature: Communication orale
  \item État d'avancement: Thèse en phase finale
  \item Resumé long: Non disponible
\end{enumerate}



\end{document}

% convert to words
% pandoc -s 0.1thesis.tex --bibliography=../shared/ref.bib -o myt.docx
