% Include preamble
\documentclass[12pt,a4paper]{article}
\usepackage[french,english]{babel}
\usepackage[T1, T2A]{fontenc}
% Load Package
% a third test
% a 5 test
\usepackage[margin=1in]{geometry} % for margin
\usepackage{blindtext}
\usepackage{graphicx}
\usepackage{multicol}
\usepackage{tabularx}
\usepackage{booktabs}
\usepackage{pgffor}
\usepackage{color}
\usepackage[hyperfootnotes=true,colorlinks=true,allcolors={blue}]{hyperref}
\usepackage{ifthen}
\usepackage{indentfirst}
\usepackage{dcolumn}

\usepackage{nameref}
\usepackage{placeins} % put this in your pre-amble
\usepackage{flafter}  % put this in your pre-amble
\usepackage{float} % to manage figure float
\usepackage{authblk} % for co-author
\usepackage{pdfpages}
\usepackage{listings} % for formating code
\usepackage{dirtytalk} % for inserting quotation
% Control footnotes spacing
%\usepackage{footnote}
%\usepackage[multiple]{footmisc}
\setlength{\footnotesep}{0.3cm}%
\setlength{\skip\footins}{0.7cm}

\usepackage{bookmark}
%\usepackage{lmodern}
%\usepackage{mathptmx} % for nice font
\fontfamily{cmss}

\hypersetup{
  colorlinks=true,
  allcolors=blue
}

\usepackage{threeparttable}
\usepackage{booktabs, longtable}
\usepackage{tabularx}
\usepackage[singlelinecheck=false]{caption}
% for source under image
\newcommand{\source}[1]{\caption*{Source: {#1}} }

% set space interligne etc
\usepackage{setspace}
\onehalfspacing




%\usepackage{amssymb} %maths
%\usepackage{amsmath} %maths
\usepackage{siunitx} % using plus-minus instead of std
\usepackage[utf8]{inputenc} %useful to type directly diacritic characters
\usepackage[autostyle]{csquotes}
\DeclareUnicodeCharacter{2212}{-} % Unicode U+2212 is MINUS SIGN, which is not set up by default with inputenc
\DeclareUnicodeCharacter{2009}{~} % Unicode U+2212 is MINUS SIGN, which is not set up by default with inputenc{00A0}{~}

\DeclareUnicodeCharacter{0080}{+} % Unicode U+2212 is MINUS SIGN, which is not set up by default with inputenc{00A0}{~}

\DeclareUnicodeCharacter{0301}{-} % Unicode U+2212 is MINUS SIGN, which is not set up by default with inputenc{00A0}{~}

\DeclareUnicodeCharacter{0300}{-} % Unicode U+2212 is MINUS SIGN, which is not set up by default with inputenc{00A0}{~}

\DeclareUnicodeCharacter{001B}{-} % Unicode U+2212 is MINUS SIGN, which is not set up by default with inputenc{00A0}{~}

% Count of words
\usepackage{moreverb} % for verbatim ouput
\immediate\write18{texcount  -inc -merge -nobib -sum -sub=section \jobname.tex > ./texcount.tex}
\newcommand\wordcount{\verbatiminput{./texcount.tex}}

% Include csv files

% normal header stat here
\title{Fiscal Impact of Immigration in Canada: A National Transfer Accounting Approach}
\author{}
\date{}

\begin{document}
% remove extra space btw table caption and table
%\captionsetup[table]{skip=0pt}
% Hide the default abstract title
%\renewcommand{\abstractname}{\vspace{-\baselineskip}}
\maketitle
\begin{abstract}
  Population aging has become the intersection of heated debates in advanced economies. One of the fierce is the role of immigration as source of labor supply and as a possible solution to alleviate the financial pressure of population aging on social protection. While the impact of immigration on the labor market has been extensively researched in Canada, less attention has been paid to its fiscal consequences.

  \vspace{0.7em}\par
  In this study, we extend the National Transfer Accounts (NTA) methodology to measure the allocation of public transfer inflows and taxes to the State by ages, for immigrants and natives. NTA integrate micro and macro database to measure the ways individuals of various ages produce, consume, save, and share resources. This country-based methodology has been applied to seventy countries around the world and this paper is the first to provide a theoretical foundation of NTA for immigrants and natives in Canada covering 1997-2016.

\end{abstract}

  \section*{Introduction}\label{sec:into}
  Immigration is costly for receiving countries: that's the message echoed in borjas's latest book: Immigration Economics, the 30 years summary of the author's work in the field of immigration\citep{Card:2016ku}. Indeed, many are those who perceive immigrants as a burden to the taxpayers and empirical studies supporting this view are common. For example, \citet{Chojnicki:2011vu} found that even though the long term effect of immigration on the French public finances is slightly positive, the life cycle average of immigration net contribution is negative for the year 2005. \citet{Fehr:2003gq} stated that, even doubling the number of immigrants, an extreme measure by most policy standards, will do little to mitigate the upcoming financial pressure in developed countries.

  \vspace{0.7em}\par
  Despite its cost, immigrant intake has been increasing in most developed country over the last decades. Apparently, \citet{Borjas:2014hr} and others present one side of the story of the which another side is that skilled migrants make a large fiscal contribution, and unskilled migrants may be net contributors if they eventually depart and make few claims on government expenditure while in the country \citep{Rowthorn:2008kk}. For example \citet{Storesletten:2000cn} found that selective immigration policies, involving an increased inflow of working-age high and medium-skilled immigrants, can remove the need for future fiscal reform. For instance, an annual intake of 1.6 million (an increase from 0.44 to 0.62 of the population) immigrants would be equivalent to alternate policy to increase tax revenue by 4.4 percentage points in the US. \citet{Akin:2012gh} and \citet{Dustmann:2014dr} also provided strong evidences that immigrants especially recent ones, has made substantial contribution to public finances.


  \vspace{0.7em}\par
  For Canada, very little is known about the net contribution of immigrants to the public finances. In one hand, recent studies about the impact of immigration have mostly addressed, the effect on the labor market and the overall welfare of the economy. For example, \citet{Ileri:2019hf} found that skilled immigration lower wages inequality and contributes positively to the overall welfare in the economy. The simulations from \citet{Dungan:2013jp} indicate that additional immigration is likely to have a positive impact on the Canadian economy. On the other hand, the rare studies on the fiscal aspect reached conflicting results. For instance, \citet{Grubel:2012wo} found that in the fiscal year 2005/2006 the average immigrant costed \$6,051,  while \citet{Javdani:2013gu} reported about \$500; and even this small impact resulted from the degradation in the composition and income attainment of immigrants over time. Indeed, using data from the 1981 census, \citet{Akbari:1989fh} found a positive net fiscal transfer of \$500; while more recently, \citet{Hering:2010tz} suggest that increasing immigrants intake rather than the retirement age, would significantly improve the fiscal sustainability of the CPP and largely solve the financing problems of the QPP.

  \vspace{0.7em}\par
  The reasons underlining these conflicting results are of twofold. First, different studies use different immigrant cohorts and assumptions about consumption of public goods \citep{Grubel:2012wo}. Second, only costs and contributions to the state that are directly related to the individual are included while those from and through the family are left out \citep{dAlbis:2019de}. In this study we use the National Transfer Account method which, take an intergenerational approach and allow to account for costs and contributions that arise through the family as well as those to the state, over a relatively large number of cohorts.

  \section*{Methods \& Data Sources}\label{sec:data}

  Various methods has been proposed for evaluating fiscal impact of immigration, including the overlapping generations model(OLG) \citep{Akin:2012gh,Ileri:2019hf,Storesletten:2000cn}, the generational accounting \citep{Auerbach:2000wg,Preston:2014uw,Sokhna:2018ww}, and more recently, the National Transfer Accounts  method proposed by \citet{Mason:2011wc}.

  \vspace{0.7em}\par
  The NTA method introduced age into the System of National Accounts (SNA) by disaggregating national income, consumption, and savings by age and therefore to take into account intergenerational transfers made by the State or the family. NTAs are based the premise that the lifecycle deficit, that is the difference between consumption and labor income by age, corresponds to the aggregate net transfers, public and private. This identity creates the links between public finances as reported in the SNA and the household economy where most of the intergenerational transfers take place. One advantage of the NTA method over the others is that by using as many years as possible of cross-sectional data, it help to estimate present expected values of lifetime net payments without relying on too much sensitive assumptions as required by other methods \citep{dAlbis:2019de}.

  \vspace{0.7em}\par
  The current NTA framework as discussed in \citet{UnitedNations:2013vz}allows the isolation of the contribution the State(among others) in funding the lifecycle deficit of an individual. This article goes further by disaggregating the State contribution by immigration status. Doing so, we compute net public transfers over the lifecycle for immigrants and natives for each year between 1997 and 2016, using data from various consumers surveys, while being consistent with national accounts and population estimates. Surveys data include the Survey of Labor and Income Dynamics (SLID) from 1997 to 2011, the Canadian Income Survey (CIS) from 2011 to 2016 and the Survey of Household Spending (SHS) from 1997 to 2016.

  \section*{Expected Results}\label{sec:results}
  We might expect a net positive life cycle effect (taxes contributed minus transfer received) of immigration on public finances, especially for recent immigrants for which economic factors has motivate the selection. Indeed, taking the complete life cycle into account, most immigrants arrive in the host Canada during a period of the lifecycle where production surpasses greatly consumption. Therefore, immigration represent a saving to taxpayer as they do not have to pay the costs of childcare and education before immigrants arrive. For example, \citet{Dustmann:2014dr} found that between 1995 and 2011 European and non-European immigrants endowed the UK labor market with human capital that would have cost £14 and £35 billion respectively, if it were produced through the British education system. Furthermore, some immigrants will return to their country of origin to spend the last and most cost intensive in health care \citep{Bratsberg:2014cl}

  \newpage
  \section*{References}
    \printbibliography[heading=none]

\end{document}