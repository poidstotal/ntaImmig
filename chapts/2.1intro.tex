Migration has always been a hot topic and one of the most controversial in industrialized countries \citep{Marois:2020je}.
More often than not, and especially during the second half of the 20th century, studies on immigration have focused on its effect on the labour market outcome of native workers \citep{Piche:2013ir,Fusaro:2018wi,Castles:2012vr}.
Although these considerations have continued till the current decade, they have somehow faded into the background of a new threat: population ageing, the increasing share of older persons in a population due to declining fertility rates and rising life expectancy.

\vspace{0.7em}\par
Population ageing has become a dominant policy concern in advanced economies for its far-reaching effects on the labour market and public finances.
It directly reduces the growth of the labour supply and increases the risk of a labour shortage.
In Canada, between 2007 to 2016, the population aged 15 and older increased by 3.1 million people, but the number of labour market participants increased only by 1.6 million \citep{Fields:2017wa}.
As the relative (to the population size) number of labour market participants decreases, government expenditures (relative to public revenue) increase, all else remaining equal.

\vspace{0.7em}\par
These prospects put heavy pressure on public finances and call for complex policies \citep{Godbout.2012,St-Maurice.2018uv,Lee:bKjc_XK_,Zokalj:2016bq,Kudrna:2015dr,Müller.2009}.
For instance, \citet{St-Maurice.2018uv} estimate that the province of Quebec would see its budgetary balance decrease from 0.2\% of GDP in 2022 to -2.8\% in 2058. This would mainly result from an increase in health expenditure, from 39.5\% of government revenues in 2022 to 62.9\% in 2058.
For the United States, \citet{Lee:bKjc_XK_} calculated that population ageing would raise the tax costs of the current benefits package by about 50\%, even with no changes in the per-recipient costs of programs.
Budgetary projections in Europe anticipate an increase of public health care expenditures in all countries by an average of 24.07\% in 2060, compared to the level in 2013 \citep{Zokalj:2016bq}.
\citet{Kudrna:2015dr} suggested that to finance the significant increase in old-age related government expenditure programs, the Australian government would have to cut non-age related expenditures by 32\% or increase the consumption tax rate by 28\% for balancing the government budget by 2050.

\vspace{0.7em}\par
As the pressure brought by population ageing on various aspects of the economy builds up, immigration increasingly appears as a source of additional labour supply and a possible solution to alleviate the pressure on public finances.
For this reason, recent decades have seen a subtle but significant change within the immigration debate from policies that harden \textit{undesirable} immigration to policies that welcome \textit{selected} immigrants.
Indeed, population ageing has given new vitality to the immigration debate. However, while selected immigrants are tailored to and absorbed by the labour market, their costs to taxpayers have been less documented \citep{Dustmann:2007fl}.

\vspace{0.7em}\par
Public opinion on immigration has traditionally been negative. Most people believe that immigrants do not pay their fair share to the tax system or receive more than they contribute to public finances.
A 2008 European Social Survey reveals that 44\% of European citizens responded that immigrants receive more than they contribute, with only 15\% believing that they receive less \citep{Dustmann:2014dr}.
Much empirical research also supports the idea that immigration is costly for receiving countries.
This message is at the core of Borjas's latest book, Immigration Economics, the 30 years summary of the author's work in the field of immigration \citep{Card:2016ku}.
In Canada, \citet{Grubel:2012wo} found that in the fiscal year 2005/2006, the average immigrant costed \$6,051, while \citet{Javdani:2013gu} reported about \$500.
Outside Canada, \citet{Chojnicki:2011vu} found that even though the long-term effect of immigration on the French public finances is slightly positive, the life cycle net contribution is negative for 2005.
\citet{Fehr:2003gq} stated that even doubling the number of immigrants, an extreme measure by most policy standards, will do little to mitigate the upcoming financial pressure in developed countries.

\vspace{0.7em}\par
As the immigration debate continues, so does immigrant intake in most developed countries \citep{Card:2016ku}.
In Canada, for example, the number of landed immigrants has remained relatively high since the early 1990s, with an average of approximately 235,000 new immigrants per year \citep{StatistiqueCanada:2016ud}.
In 2017, the country welcomed more than 286,000 permanent residents. Still, the government adopted a historical multi-year plan to grow its annual immigration levels to 340,000 by 2020 (2018 Annual Report to Parliament on Immigration).
These policies suggest that, contrary to \citet{Borjas:2014hr} and others, skilled migrants significantly contribute to public finances. Even unskilled migrants may be net contributors if they eventually depart or make few claims on government expenditures while in the country \citep{Rowthorn:2008kk}.
\citet{Akbari:1989fh} found a positive net fiscal transfer of \$500 using data from the Canadian census in 1981, while results from \citet{Ileri:2019hf} and \citet{Dungan:2013jp} also suggest that immigration is likely to positively impact the Canadian economy, including lowering wages inequality and improving overall welfare.

\vspace{0.7em}\par
In the US, \citet{Storesletten:2000cn} found that selective immigration policies involving an increased inflow of working-age high and medium-skilled immigrants can remove the need for future fiscal reform.
For instance, an annual intake of 1.6 million immigrants (an increase from 0.44\% to 0.62\% of the population) would be equivalent to an alternate policy to increase tax revenue by 4.4 percentage points in the US.
\citet{Akin:2012gh} for Germany and \citet{Dustmann:2014dr} for the United Kingdom also provide strong evidence that immigrants, especially recent ones, have made substantial contributions to public finances.

\vspace{0.7em}\par
Despite being intensely debated as a policy response to population ageing, immigration has yet to receive enough fiscal analysis to support current policies.
The reasons are twofold.
First, different studies make different assumptions about the consumption of public goods \citep{Grubel:2012wo} and most studies only account for costs and contributions that are directly related to the individual, while those from and through the family are left out \citep{dAlbis:2019de}.
Second, the scope of the immigrant population is not consistent across studies, and results vary for different cohorts \citep{Grubel:2012wo}, subgroups and methodology \citep{Chojnicki:2011vu}.
As an illustration, \citet{Lee:1998fs} found that the overall fiscal impact (taxes paid minus costs generated) is, on average, \$1,400 for first-generation immigrants, -\$400 for first and second generations, and \$600 if extended to all descendants of living immigrants.

\vspace{0.7em}\par
Not only have \citet{Lee:1998fs} distinguished the fiscal impact of immigration by the generation of immigrants, but also they have included many costs and benefits, including public goods.
Such a comprehensive approach is infrequent and almost nonexistent for Canada.
This study fills the gap by using the National Transfer Account (NTA) method to measure the costs and contributions of immigration between 1997 and 2015.
The NTA method takes an intergenerational perspective that accounts for costs and contributions involving the family and the state \citep{Mason:2011wc,UnitedNations:2013vz}.
This article builds on \citet{Merette:2019kz}, splits inflow and outflow transfers between immigrants and natives, measures the differences between the two populations, and attempts to uncover the sources of these differences using demographic decomposition.

