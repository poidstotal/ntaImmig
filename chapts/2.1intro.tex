Migration has always been a very hot topic and one of the most controverted in industrialized countries \citep{Marois:2020je}.
Traditionally, the immigration debate was mostly focused on policy to prevent foreigners invasion of the labor market which may lead to the degradation of employment opportunity for natives \citep{Fusaro:2018wi} and the loss of national identity \citep{Castles:2012vr}.
For this reason, studies during the second half of the 20th century have largely focused on the effect of immigration on the labor market outcome on native workers \citep{Piche:2013ir}.
Although these considerations have continued till the current decade, they have somehow faded in the background of a new threat: population aging, an increasing share of older persons in a population due to rising life expectancy and declining fertility rates.

\vspace{0.7em}\par
Population aging has become a dominant policy concerns in advanced economies for its far-reaching effects on the labor market and public finances.
It directly reduces the growth of the labor supply and increases the risk of a labor shortage.
In Canada between 2007 to 2016, the working-age population (aged 15 and older) increased by 3.1 million people, but the number of labor market participants increased only by 1.6 million \citep{Fields:2017wa}.
As the number of labor market participants decreases in relation to the population, government expenditures increase in relation to public revenue, all else remaining equal.
For the United States, \citet{Lee:bKjc_XK_} calculated that population aging will raise the tax costs of current benefit package by about 50\%, even with no changes in the per-recipient costs of programs.
Budgetary projections in Europe anticipate an increase of public health care expenditures in all countries by an average of 24.07\% in 2060, compare to the level in 2013 \citep{Zokalj:2016bq}.
These prospects put a heavy pressure on public finances and call for difficult policies.
For example, \citet{Kudrna:2015dr} suggested that, in order to finance the significant increase in old-age related government expenditure programs, the Australian government would have to either cut non-age related expenditures by 32\% or increase consumption tax rate by 28\% for the government budget to be balanced in 2050.

\vspace{0.7em}\par
As the pressure brought by population aging on various aspects of the economy builds up, immigration is increasingly being looked upon not only as a source of additional labor supply but also as a possible solution to alleviate the pressure on public finances.
For this reason, recent decades have seen a subtle but significant change within the immigration debate from policies that harden "undesirable" immigration to policies that welcome "selected" immigrants.
Indeed, population aging has given a new vitality to the immigration debate, but while selected immigrants are tailored to, and absorbed by the labor market, their costs to taxpayers, a more important concern has been less documented \citep{Dustmann:2007fl}.

\vspace{0.7em}\par
Public opinion on immigration has traditionally been negative with most people believing that immigrants do not pay their fair share to the tax system or receive more than they contribute to public finances.
A 2008 European Social Survey reveals that 44\% of European citizens responded that immigrants receive more than they contribute, with only 15\% believing that they receive less \citep{Dustmann:2014dr}.
Much empirical research also supports the idea that immigration is costly for receiving countries.
This message is echoed in Borjas's latest book, Immigration Economics, the 30 years summary of the author's work in the field of immigration \citep{Card:2016ku}.
In Canada, \citet{Grubel:2012wo} found that in the fiscal year 2005/2006, the average immigrant costed \$6,051, while \citet{Javdani:2013gu} reported about \$500.
Outside Canada, \citet{Chojnicki:2011vu} found that even though the long-term effect of immigration on the French public finances is slightly positive, the life cycle net contribution is negative for the year 2005.
\citet{Fehr:2003gq} stated that even doubling the number of immigrants, an extreme measure by most policy standards, will do little to mitigate the upcoming financial pressure in developed countries.

\vspace{0.7em}\par
While the immigration debates continue, immigrants intake has been increasing in most developed countries \citep{Card:2016ku}.
In Canada for example, the number of landed immigrants has remained relatively high since the early 1990s, with an average of approximately 235,000 new immigrants per year \citep{StatistiqueCanada:2016ud}.
In 2017, the country welcomed more than 286,000 permanent residents and the government adopted a historic multi-year levels plan to grow its annual immigration levels to 340,000 by 2020 (2018 Annual Report to Parliament on Immigration).
This suggests that \citet{Borjas:2014hr} and others present one side of the story of which the other side is that skilled migrants make a large fiscal contribution, and unskilled migrants may be net contributors if they eventually depart or make few claims on government expenditures while in the country \citep{Rowthorn:2008kk}.
\citet{Akbari:1989fh} found a positive net fiscal transfer of \$500 using data from the Canadian census in 1981 while \citet{Hering:2010tz} suggest that increasing immigrants intake rather than the retirement age, would significantly improve the fiscal sustainability of the CPP (Canada Pension Plan) and largely solve the financing problems of the QPP (Quebec Pension Plan).
Results from \citet{Ileri:2019hf} and \citet{Dungan:2013jp} also suggest that immigration is likely to have a positive impact on the Canadian economy including the lowering of wages inequality and improvement of overall welfare.

\vspace{0.7em}\par
In the US, \citet{Storesletten:2000cn} found that selective immigration policies, involving an increased inflow of working-age high and medium-skilled immigrants, can remove the need for future fiscal reform.
For instance, an annual intake of 1.6 million (an increase from 0.44\% to 0.62\% of the population) immigrants would be equivalent to an alternate policy to increase tax revenue by 4.4 percentage points in the US.
\citet{Akin:2012gh} for Germany and \citet{Dustmann:2014dr} for the United Kingdom, also provides strong evidence that immigrants especially recent ones, has made a substantial contribution to public finances.

\vspace{0.7em}\par
Although immigration is highly debated in the context of population aging and its fiscal impacts are attracting increasing attention, the literature has yet to produce non-partisan results to support current immigration policies in advanced economies and Canada in particular.
The reasons are twofold.
First, different studies make different assumptions about the consumption of public goods \citep{Grubel:2012wo} and most studies only account for costs and contributions that are directly related to the individual while those from and through the family are left out \citep{dAlbis:2019de}.
Second, the scope of the immigrant population is not consistent across studies and results varies for different cohorts \citep{Grubel:2012wo}, subgroup and methodology \citep{Chojnicki:2011vu}.
This is illustrated in \citet{Lee:1998fs} who found that the overall fiscal impact (taxes paid less costs generated) is on average, \$1,400 if only first-generation immigrants are included, -\$400 if the second generation is included, and  \$600 if extended to all descendants of living immigrants.
Such a holistic approach is very rare and almost nonexistent for Canada.
In this study, we use the National Transfer Account (NTA) method to measure the costs and contributions of immigration between 1997 and 2015.
The NTA method takes an intergenerational perspective that accounts for costs and contributions involving the family and the state \citep{Mason:2011wc,UnitedNations:2013vz}.
This article builds on \citet{Merette:2019kz}, splits inflow and outflow transfers between immigrants and natives, measures the differences between the two populations, and attempts to uncover the sources of these differences using demographic decomposition.

