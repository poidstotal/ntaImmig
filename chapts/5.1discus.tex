\subsection{Source of income gap}
The income gap between immigrants and natives arises from several factors in the labour market, including the decision to participate, hiring discrimination, and wage discrimination.

\vspace{0.7em}\par
The decision to participate in the labour market partially relates to the individual or household characteristics.
For example, women would prefer taking care of their young child, perhaps for financial or cultural\footnote{coming from culture where babysitting or professional childcare is not well ingrained} reasons, instead of using professional babysitting or public childcare service.
Some immigrants, both men and women, are constrained to going back to school just after their arrival or after facing difficulties in the labour market.
As a result, immigrants aged 25-54 and who landed five or fewer years earlier are significantly less active in the labour market than their counterpart natives, with only 77.4\% against 88.8\% for counterpart natives in 2019 (Statistics Canada: Table 14-10-0083-01).
After deciding to participate in the labour market, immigrants may also have to overcome hiring discrimination when finding and keeping a job.

\vspace{0.7em}\par
Hiring discriminations consist of barriers that make it difficult for immigrants to find and retain adequate jobs.
For example, immigrants have more difficulties making it to the interview stage, as recruiters dismiss their application if their name is foreign-sounding \citep{Crandall:wo,ROOTH2010523,Oreopoulos:2011jv}.
Directly, hiring discriminations result in a higher level of unemployment among recent immigrants compared to established immigrants and natives \citep{Oreopoulos:2011jv}.
For instance, in 2019, the unemployment rate among immigrants that landed five years or less was 9.9\% compared to 6.5\% for those who landed more than 5 to 10 years earlier and 5.5\% for natives \citep{statCan:002}.
Indirectly, hiring discriminations have two consequences.

\vspace{0.7em}\par
First, they constrain immigrants to take temporary or seasonal positions and thus to work fewer hours than desired.
This situation is referred to as underemployment \citep{ilo:2013icls,gilbert:2022b,Mitchell:2008wo,CanadianLabourCongress:2014wi}.
\citet{gilbert:2022b} estimate that between 2013 and 2017, the odds of being underemployed for recent immigrants compared to natives is about 40\% higher among women and 20\% for men.
Even ten years after arrival, the average immigrant still has 15\% more unmet desire of hours than natives.

\vspace{0.7em}\par
Second, hiring discriminations push immigrants toward jobs for which they are massively overqualified.
\citet{Li:2006uu} found that immigrants are among the most susceptible to being over-qualified, as 52\% (compared to only 28\% among their Canadian-born counterparts) of recent immigrants with a university degree have worked in a job requiring only high school education at some point between 1993 and 2001.

\vspace{0.7em}\par
Underemployment and overqualification have resulted in the rise of low-income rates, which is the percentage of persons living under the poverty line, for each successive cohort of immigrants between 1980 and 2000  \citep{picot2003rise}
For example, among immigrants with a university degree and working full-time, over-qualified men earned 42\% per week in 2000 less than their counterparts employed in jobs requiring a university degree.
The gap was 39\% for women and 47\% for young men \citep{Morissette:wh}.

\vspace{0.7em}\par
If hiring discriminations contribute to low income among immigrants, it accounts for a relatively small part of the earning gap between immigrants and Canadian-born.
For instance, \citet{Morissette:wh} found that after controlling for the highest education and job required degrees, immigrants working full-time still earned about 20\% less than natives in 2000, and the gap persists even beyond ten years after arrival.
This result suggests that the persisting difference between immigrants and natives is due, at least in part, to wage discrimination, even for the same education and job position.

\vspace{0.7em}\par
Wage discrimination arises when immigrants receive lower payments than natives for the same type and amount of work.
One reason for such discrimination is that employers attach a lower value to education and work experiences from some countries than others\citep{Coulombe:2014ir,Fortin:2016hl}.
Employment and earning discriminations go, however, beyond educational considerations.
\citet{Block:2019va} see these as the tip of the iceberg of unequivocal racialized economic discrimination in Canada.
As a result, non-racialized immigrants do better in the Canadian labour market and sooner than racialized immigrants.
Moreover, income inequality between racialized and non-racialized Canadians extends to second generations and beyond \citep{Block:2019va,Nadeau:2010jd,Hum:2000gz}

\subsection{Limitations and future research}
This study contributes new results to the immigration debates using new datasets and advanced methods.
However, there is room for improvement in various areas, including the effects of a changing demographic structure, extended scope of the immigrant population, the age at arrival, and the healthy immigrant effect.
As we have seen, omitting the demographic difference between immigrants and natives results in significant bias in analyzing the transfer differential between the two populations.

\vspace{0.7em}\par
This study addressed the bias that would result from a difference in the age structure between immigrant and native populations. However, the changes in the age structure from one year to another may also introduce bias in the trend comparison. The bias may be slight for consecutive years but significant over many years as the population ages and immigration continues.
The logic for such bias is the same that justifies accounting for inflation when comparing the price difference of two baskets of products over time.

\vspace{0.7em}\par
Demographic effects may also arise from a different composition of the immigrant population.
For instance, although this study has gathered data over many years, the cross-sectional nature of these data makes it applicable only to the first generation of immigrants.
As pointed out by previous studies \citep{Lee:1998fs}, defining the immigrant population is particularly challenging, and enlarging the immigrant population by including more generations may lead to different results.
The first generation refers to people born outside Canada but now residing as citizens or permanent residents.
Those born in Canada but have at least one parent born outside the country belong to the second generation, while those with both parents and themselves born in Canada belong to the third generation.
With such variants in the immigrant population, their effect on the Immigrant Surplus would be worth investigating.

\vspace{0.7em}\par
Even for the first generation of immigrants, the age at arrival could be a source of difference in transfers.
For instance, there is a general assumption that immigrants arriving at working age represent a saving in childhood and education expenses which primarily occur in the country of origin.
For example, \citet{Dustmann:2014dr} found that between 1995 and 2011, European and non-European immigrants endowed the UK labour market with human capital that would have cost \pounds14 and \pounds35 billion respectively if produced through the British education system.
Unfortunately, this study has not accounted for the age of arrival.
However, the results suggest that immigrants have made a similar contribution in Canada, as they represent about 24.2\% of the Canadian population but are responsible for only 14.5\% of education costs in 2015.

\vspace{0.7em}\par
If arriving later implies saving in education costs for the host country, departing earlier is also expected to reduce age-related expenses.
Health care, for example, could see some savings if some immigrants return to their home country to spend the last and most cost-intensive part of their life \citep{Bratsberg:2014cl}.
This expectation does not seem to apply in Canada, as our research shows that immigrants accounted for 24.2\% of the population but 29.5\% of health care expenses in 2015. This result may be related to using total medical consultation as a proxy for health care expenses and, therefore, will require further investigation, especially given the other expectation that immigrants are healthier than natives \citep{Ichou:2019ik,Vang:2016di}.
This foreign-born health advantage is known as the Healthy Immigrant Effect (HIE). It is mainly the result of immigration policies that put a relatively high weight on health status and disqualify applicants whose health conditions would cause excessive demand for health care or social services.


\subsection{Conclusion}

Overall, the average immigrant contributed for about \$\numUp{AveImmigrantsOutflows} per year while receiving about \$\numUp{AveImmigrantsInflows} per year on average between 1997 and 2017.
A native on the other hand has contributed \$\numUp{AveNativesOutflows} but received \$\numUp{AveNativesInflows}.
In net, immigrants received about \$\numUp{AveperCapita} more than natives on average between 1997 and 2015 and this surplus is increased to \$\numUp{AveAdj.Transfer} when comparing immigrants and natives at the same age.
Labour market imbalances are the primary sources of this difference, accounting for \numUp{SIprop}\% of the Net Surplus.

\vspace{0.7em}\par
These results lie somewhere between the results from \citet{Grubel:2012wo} and \citet{Javdani:2013gu} who reported \$\num{6051} and \$500 respectively for the fiscal year 2005/2006.
Although reducing this divergence alone contributes to a healthier immigration debate, a more significant contribution from this study lies in using the NTA method, which is more englobing than previous methods regarding the type of transfers included.
Moreover, the striping out of demographic disturbances through effect decomposition has brought to the light effects that would stay hidden otherwise.
These methods clearly show the cost of immigration to public finances and point to labour market imbalances as the primary source of this cost.
Therefore, rather than debating whether or not immigrants' intake should be increased or reduced, it would be beneficial to debate how to enable immigrants to achieve their full potential in the labour market.
The solution to such questions may involve adjusting the selection criteria. However, it indeed calls for addressing the labour market imbalances.

\vspace{0.7em}\par
While this research and the extensions suggested would provide insights into the fiscal effect of immigration, a more fundamental investigation into the nature and causes of socioeconomic inequality is required to reduce these differences.
Canada and the world will need agile thoughts leaders and decision-makers who can assess the building block of our society and engage the changes required for its improvement.
In that direction, a starting but critical point in Canada would be to investigate and address the sudden degradation in the Immigrant Surplus from 2011.