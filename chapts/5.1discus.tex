

\subsection{Improvements for future research}
This study contributes new results to the immigration debates using new datasets and advanced methods.
However, there is room for improvement in various areas, including the effects of a changing demographic structure, extended scope of the immigrant population, the age at arrival, and the healthy immigrant effect.
As we have seen, omitting the demographic difference between immigrants and natives results in significant bias in analyzing the transfer differential between the two populations.

\vspace{0.7em}\par
This study addressed the bias that would result from a difference in the age structure between immigrant and native populations. However, the changes in the age structure from one year to another may also introduce bias in the trend comparison. The bias may be slight for consecutive years but significant over many years as the population ages and immigration continues.
The logic for such bias is the same that justifies accounting for inflation when comparing the price difference of two baskets of products over time.

\vspace{0.7em}\par
Demographic effects may also arise from a different composition of the immigrant population.
For instance, although this study has gathered data over many years, the cross-sectional nature of these data makes it applicable only to the first generation of immigrants.
As pointed out by previous studies \citep{Lee:1998fs}, defining the immigrant population is particularly challenging, and enlarging the immigrant population by including more generations may lead to different results.
The first generation refers to people born outside Canada but now residing as citizens or permanent residents.
Those born in Canada but have at least one parent born outside the country belong to the second generation, while those with both parents and themselves born in Canada belong to the third generation.
With such variants in the immigrant population, their effect on the Immigrant Surplus would be worth investigating.

\vspace{0.7em}\par
Even for the first generation of immigrants, the age at arrival could be a source of difference in transfers.
For instance, there is a general assumption that immigrants arriving at working age represent a saving in childhood and education expenses which primarily occur in the country of origin.
For example, \citet{Dustmann:2014dr} found that between 1995 and 2011, European and non-European immigrants endowed the UK labour market with human capital that would have cost \pounds14 and \pounds35 billion respectively if produced through the British education system.
Unfortunately, this study has not accounted for the age of arrival.
However, the results suggest that immigrants have made a similar contribution in Canada, as they represent about 24.2\% of the Canadian population but are responsible for only 14.5\% of education costs in 2015.

\vspace{0.7em}\par
If arriving later implies saving in education costs for the host country, departing earlier is also expected to reduce age-related expenses.
Health care, for example, could see some savings if some immigrants return to their home country to spend the last and most cost-intensive part of their life \citep{Bratsberg:2014cl}.
This expectation does not seem to apply in Canada, as our research shows that immigrants accounted for 24.2\% of the population but 29.5\% of health care expenses in 2015. This result may be related to using total medical consultation as a proxy for health care expenses and, therefore, will require further investigation, especially given the other expectation that immigrants are healthier than natives \citep{Ichou:2019ik,Vang:2016di}.
This foreign-born health advantage is known as the Healthy Immigrant Effect (HIE). It is mainly the result of immigration policies that put a relatively high weight on health status and disqualify applicants whose health conditions would cause excessive demand for health care or social services.

\subsection{Conclusion}

Overall, the average immigrant contributed for about \$\numUp{AveImmigrantsOutflows} per year while receiving about \$\numUp{AveImmigrantsInflows} per year on average between 1997 and 2017.
A native on the other hand has contributed \$\numUp{AveNativesOutflows} but received \$\numUp{AveNativesInflows}.
In net, immigrants received about \$\numUp{AveperCapita} more than natives on average and this surplus is increased to \$\numUp{AveAdj.Transfer} when comparing immigrants and natives at the same age.
Labour market imbalances are the primary sources of this difference, accounting for \numUp{SIprop}\% of the Net Surplus.

\vspace{0.7em}\par
These results lie somewhere between the results from \citet{Grubel:2012wo} and \citet{Javdani:2013gu} who reported \$\num{6051} and \$500 respectively for the fiscal year 2005/2006.
Although reducing this divergence alone contributes to a healthier immigration debate, a more significant contribution from this study lies in using the NTA method, which is more englobing than previous methods regarding the type of transfers included.
Moreover, the striping out of demographic disturbances through effect decomposition has brought to the light effects that would stay hidden otherwise.
These methods clearly show the cost of immigration to public finances and point to labour market imbalances as the primary source of this cost.
Therefore, rather than debating whether or not immigrants' intake should be increased or reduced, it would be beneficial to debate how to enable immigrants to achieve their full potential in the labour market.
The solution to such questions may involve adjusting the selection criteria.
However, it also calls for more research and policies to address the labour market imbalances, especially the sudden degradation of the Immigrant Surplus since 2011.