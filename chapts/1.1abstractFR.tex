Au Canada, l'immigration constitue la principale réponse au vieillissement de la population. Si de nombreuses recherches ont porté sur l'impact de l'immigration sur divers aspects de l'offre de main-d'œuvre, l'aspect financier a reçu moins d'attention.
Dans cette étude, nous appliquons la méthode du compte de transfert national (NTA) et la décomposition démographique pour estimer le coût fiscal net de l'immigration au Canada entre 1997 et 2015.
Les résultats montrent qu'en moyenne, les immigrants ont reçu environ \$\numUp{AveperCapita} de plus en transfert net par habitant que les natifs entre 1997 et 2015.
Cependant, ce coût est principalement le résultat des déséquilibres du marché du travail qui, après avoir éliminé l'effet des différences démographiques, représentent \numUp{SIprop}\% de l'excédent.