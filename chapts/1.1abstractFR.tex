Le vieillissement de la population est devenu le centre de débats passionnés dans les économies avancées.
Les politiques publiques visant à atténuer ses effets sur l'offre de main-d'œuvre et les finances publiques sont de plus en plus orientées vers l'immigration.
Alors que l'impact de l'immigration sur divers aspects du marché du travail a fait l'objet de recherches approfondies, son impact budgétaire a reçu moins d'attention.
Dans cette étude, nous appliquons la méthode des Comptes de transferts nationals (NTA) et la décomposition démographique pour estimer le coût fiscal net de l'immigration au Canada entre 1997 et 2015.

\vspace{0.7em}\par
Les résultats montrent que par habitant entre 1997 et 2015, les immigrants ont reçu environ \DTLfetch{statex}{sKey}{AveperCapita}{sVal}\$ de plus que les personnes nées au Canada dans le cadre des transferts publics.
Cependant, le fait de tenir compte de la différence de structure d'âge entre les deux populations augmente le surplus total à \DTLfetch{statex}{sKey}{AveAdj.Transfer}{sVal}\$ et révèle que les déséquilibres du marché du travail représentent \DTLfetch{statex}{sKey}{SIprop}{sVal}\% du surplus.

Cette étude contribue aux connaissances pour soutenir la prise de décision en matière de politiques publiques dans le contexte du vieillissement de la population tout en améliorant la compréhension des chercheurs, principalement économistes et démographes travaillant dans le domaine des transferts intergénérationnels sur les sources d'inégalité entre immigrants et natifs.